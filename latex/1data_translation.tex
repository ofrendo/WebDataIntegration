\subsection{Data collection}
The results of our data collection are shown in table 1, which describes the basic profiles of each dataset. 

%To sum up the results of our data collection
%From these four datasets we could get the following data, describing the basic information about four datasets.

%For each subsubsection below:
%Which format did the data have? CSV, JSON, XLSX?
%Size of each dataset? (how many entities)
%Which attributes were collected for each dataset? 
%Why? Optional ones? Nonoptional ones? Resulting sparsity of each attribute?
%How did collect each dataset? Queries (give an excerpt), downloads...






\begin{table}[H]
\centering
\begin{tabularx}{\linewidth}{|X|l|l|l|l|l|}
\hline
 & Source & Format & Class & \#Entities & \#Attributes \\ \hline
 & \multicolumn{5}{c|}{List of attributes} \\ \hline
\multirow{2}{*}{Forbes} & forbes.com & xlsx & \multicolumn{1}{l|}{company} & \multicolumn{1}{c|}{2000} & \multicolumn{1}{c|}{7} \\ \cline{2-6} 
 & \multicolumn{5}{l|}{\begin{tabular}[c]{@{}l@{}}name, countries, industries, revenue, assets,\\ marketValue, profit\end{tabular}} \\ \hline
\multirow{2}{*}{DBpedia} & dbpedia.org/sparql & csv & company & \multicolumn{1}{c|}{16051} & \multicolumn{1}{c|}{9} \\ \cline{2-6} 
 & \multicolumn{5}{l|}{\begin{tabular}[c]{@{}l@{}}name, countries, industries, revenue, numberOfEmployees,\\ dateFounded, profit, keyPeople, locations\end{tabular}} \\ \hline
\multirow{2}{*}{Freebase} & freebase.com/query & json & company & \multicolumn{1}{c|}{3182} & \multicolumn{1}{c|}{9} \\ \cline{2-6} 
 & \multicolumn{5}{l|}{\begin{tabular}[c]{@{}l@{}}name, countries, industries, revenue, numberOfEmployees,\\ dateFounded, profit, keyPeople, locations\end{tabular}} \\ \hline
\multirow{2}{*}{DBpedia} & dbpedia.org/sparql & csv & location & \multicolumn{1}{c|}{3270} & \multicolumn{1}{c|}{5} \\ \cline{2-6} 
 & \multicolumn{5}{l|}{locationName, country, population, area, elevation} \\ \hline
\end{tabularx}
\caption{Basic profile of each dataset}
\label{my-label}
%\end{center}
\end{table}




\subsection{Integrated schema}
%Take from data\ForbesPlan\Forbes-integrated schema.docx
We looked into four datasets and did the following Integrated Schema.In this table we use prefix dataset 1, 2, 3,4 respectively represent Forbes, DBpedia(company), Freebase and DBpedia(Location)

%================ BEGIN TABLE

\begin{table}[H]
\centering
\begin{tabularx}{\linewidth}{|X|c|c|}
\hline
\textbf{Class Name} & \textbf{Attributes Name} & \textbf{\begin{tabular}[c]{@{}c@{}}Datasets in which\\ the attribute is found\end{tabular}} \\ \hline
company & (company) name & dataset 1, 2, 3 \\ \hline
company, location & country & dataset 1, 2, 3, 4 \\ \hline
company & industries & dataset 1, 2, 3 \\ \hline
company & revenue & dataset 1, 2, 3 \\ \hline
company & numberOfEmployees & dataset 2, 3 \\ \hline
company & dateFounded & dataset 2, 3 \\ \hline
company & assets & dataset 1, 2 \\ \hline
company & marketValue & dataset 1 \\ \hline
company & profit & dataset 1, 3 \\ \hline
company & continent & dataset 1 \\ \hline
company & keyPeople & dataset 2, 3 \\ \hline
company, location & (location) name & dataset 2, 3, 4 \\ \hline
location & population & dataset 4 \\ \hline
location & area & dataset 4 \\ \hline
location & elevation & dataset 4 \\ \hline
\end{tabularx}
\caption{Integrated schema}
\label{my-label}
\end{table}

%==================== END TABLE




\subsection{Data transformations}
%Oliver
Transformations were applied at two different points in this phase. The first was applied during mapping while the second was applied in the Java project. To begin with, numeric attributes with big values such as revenue or assets were often retrieved in scientific notation. Accordingly a function within MapForce was used to convert the numbers into a decimal notation. Secondly the datasets did not possess an ID attribute. Because it was going to be used later, it was generated with \texttt{GenerateID} in MapForce. 

The next transformations occured in Java. Many values, especially from the two DBpedia datasets, were loaded in the form of a URL due to our SPARQL query. As such the URL was parsed and only the actual value was kept. In addition, punctation and symbols such as "\_" were removed. Lastly we normalized country values, which was an important step for the blocking functions used later on in identity resolution. Especially values for the USA were transformed from spellings such as "US", "USA", "United States" to "United States of America".



%Numeric attributes to do with money: Scientific notation, small values from Forbes
%	e.g. 1.17E+10 ==> 11.700.000.000
% 	e.g. 483,1 (Apple in Forbes) ==> 483.100.000.000
%DBpedia values: Very often URL
%Company and location names: Removed common strings that are inconsistent among datasets, e.g. Apple / Apple, Inc.
% 	also: Replaced punctuation and symbols like "_" or "."
%	Result example: Apple (Forbes) <-> Apple,_Inc (DBpedia)


%Oliver
%Which attributes did we transform in which way? 
%Also: Example for each transformation showing why we did it


%Countries: Replaced different spellings of same word 
%	e.g. US, USA, U.S., America with "United States of America" or UK, England with United Kingdom


%=================================== OLD
%\begin{table}[H]
%\centering
%\caption{Integrated Schema}
%\label{my-label}
%\begin{tabular}{|c|c|c|c|}
%\hline
%Class Name & Attributes Name & \begin{tabular}[c]{@{}c@{}}Datasets in which \\ attribute is found\end{tabular} & \begin{tabular}[c]{@{}c@{}}Conflict resolution\\        %strategies\end{tabular} \\ \hline
%company & (company)name & dataset 1, 2, 3 & \begin{tabular}[c]{@{}c@{}}LongestString, \\ FavourSources, \\ Voting\end{tabular} \\ \hline
%company,location & country & dataset 1, 2, 3, 4 & Voting \\ \hline
%company & industries & dataset 1, 2, 3 & Union \\ \hline
%company & \begin{tabular}[c]{@{}c@{}}Sales/\\ revenue\end{tabular} & dataset 1, 2, 3 & FavourSources \\ \hline
%company & \begin{tabular}[c]{@{}c@{}}number Of \\ Employees\end{tabular} & dataset 2, 3 & \begin{tabular}[c]{@{}c@{}}Average,\\ Growth: Max\end{tabular} \\ \hline
%company & \begin{tabular}[c]{@{}c@{}}founding year/ \\ date founded\end{tabular} & dataset 2, 3 & \begin{tabular}[c]{@{}c@{}}MostComplete\\ (complete date) \\ AND,\\ MostComplete \\ %(sample)\end{tabular} \\ \hline
%company & Assets & dataset 1, 2 & FavourSources \\ \hline
%company & Market Value & dataset 1 & / \\ \hline
%company & profit & dataset 1, 3 & FavourSources \\ \hline
%company & continent & dataset 1 & / \\ \hline
%company & \begin{tabular}[c]{@{}c@{}}keyPeople/\\  leadership\end{tabular} & dataset 2, 3 & Union \\ \hline
%\begin{tabular}[c]{@{}c@{}}company,\\ location\end{tabular} & name & dataset 2, 3, 4 & \begin{tabular}[c]{@{}c@{}}Union AND \\ FavourSources\end{tabular} \\ \hline
%location & \begin{tabular}[c]{@{}c@{}}population\\ total\end{tabular} & dataset 4 & / \\ \hline
%location & area total & dataset 4 & / \\ \hline
%location & elevation & dataset 4 & / \\ \hline
%\end{tabular}
%\end{table}


